\paragraph*{}
  Les fluides peuvent être modéliser à plusieurs échelles, chacune des ces description utilise des approches différentes 
  et des outils mathématiques propres.
  La description continu décrit le fluide comme étant une série de champ remplissant l'espace et est régie par les   
  équations de \NS{} ici dans le cas d'un fluide incompressible\footnote{Ce qui correspond par exemple à l'eau et à l'air 
  a température et pression usuelles.} :
  
  \begin{itemize}
    \itemb Bilan de la quantité de mouvement :
    \begin{equation}\label{eq:NS1}
      \pderiv{\u}{t} + (\u \cdot \vnabla)\u = - \dfrac{\vnabla P}{\rho} + \nu \Delta \u + \dfrac{\F}{\rho},
    \end{equation}
    
    \itemb condition sur le champ de vitesse pour un fluide incompressible :
    \begin{equation}\label{eq:NS2}
      \vnabla \cdot \u = 0.
    \end{equation}
  \end{itemize}
  
  Cette description est la description macroscopique du fluide elle se place à une échelle dans laquelle les quantités 
  varient continûment dans l'espace et permettent ainsi l'utilisation de l'analyse mathématique.
  Cette description est suffisante pour la plupart des applications, a tel point que l'on résume souvent la mécanique des
  fluides à sa description continue\footnote{Cette forme d'essentialisme et parfaitement illustré dans ce compte-rendu.}.
  
\paragraph*{}
  On peut aussi décrire les fluides comme une collection de particules, atomes et molécules, qui interagissent.
  Décrire un fluide sur la base de leurs interactions constitue les prémisses des LBM, en effet, la méthode pré-datant les
  LBM (la méthode LGA pour lattice gas automata) consistait en un réseau dans lequel pouvais interagir des particules, à 
  chaque instant une maille du réseau était soit vide ou pleine.
  
  Cette méthode est très laborieuse et possède beaucoup d'artefacts dépendant du type de maillage utilisée 
  \cite{succi2001lattice}.

\paragraph*{}
  Une dernière échelle permettant la description d'un fluide et la description mésoscopique.
  A cette échèle intermédiaire on décrits le fluide en terme de densité de probabilité $f(\r,\v, t)$ de trouver une 
  particule à une position dans l'espace $\r$, et un temps $t$, avec une vitesse $\v$.
  Cette échelle intermédiaire de description et connue sous le nom de théorie cinétique des gaz.
  Le pendant mésoscopique des équations de \NS{} est l'équation de Boltzmann
  
  \begin{equation} \label{eq:Bol}
    \pderiv{f}{t} + \v \cdot \vnabla_{\r} f + \frac{\F}{\rho} \cdot \vnabla_{\v} f = C(f).
  \end{equation}
    
  Il est possible de d'exprimer les équations de \NS{} comme une petite perturbation de la distribution de Maxwell-
  Boltzmann en utilisant le méthode de Chapman-Enskog \cite{REIDER1995459}.
  Cette dernière échelle de description est celle utilisée par les LBM et est à l'origine des propriétés uniques des LBM.