\paragraph*{}
  Nous allons nous attarder ici\footnote{\hyperlink{authors}{Les auteurs} souhaitent ici préserver la santé mentale du
  lecteur.} sur les différentes grandeurs nécessaire à la description d'un fluide ainsi que les conventions qui serons
  utilisées dans la suite du compte-rendu\footnote{Il existe autant de conventions différentes que 
  de papier sur les LBM nous nous permettrons donc de rajouter notre pierre à cette horrible tradition.}.
  
\paragraph*{}
  Commençons par Les grandeurs macroscopiques, elles sont les suivantes :
  \begin{itemize} \label{eq:defmacro}
    \itemb Le champ de vitesse $\u$ du fluide :
    $$[\u] = L \cdot T^{-1},$$
    $$\u(\r,t) : \R^{D+1} \to \R^D.$$
    \emph{Ici $D$ indique la dimension spatiale de notre problème pour la suite nous nous limiterons au cas 2D ($D$=2).}\\
    \itemb La pression interne au fluide $P$ :
    $$[P] = M\cdot L^{-1}\cdot T^{-2},$$
    $$P(\r,t) : \R^{D+1} \to \R.$$
    \emph{La pression est l'une des formes que prend l'énergie interne du fluide elle peut se convertir en énergie 
    cinétique et réciproquement.}\\
    \itemb La masse volumique du fluide $\rho$ :
    $$[\rho] = M\cdot L^{-3},$$
    $$\rho(\r,t) : \R^{D+1} \to \R^{+}.$$
    \emph{Pour un fluide incompressible, $\rho$ est une constante du fluide et ne varie pas dans le temps néanmoins, 
    pour la suite, il reste utile de le définir comme une fonction de l'espace et du temps.}\\
    \itemb La viscosité cinématique du fluide $\nu$ :
    $$[\nu] = L^{2}\cdot T^{-1},$$
    $$\nu \in \R^+_*.$$
    \emph{La viscosité cinématique est une constante du fluide et peut se comprendre comme la capacité du fluide
    <<diffuser>> sa vitesse.}\\
    \itemb Le champ de force externe $\F$ :
    $$[\F] = M\cdot L\cdot T^{-2},$$
    $$\F = 0.$$
    \emph{Dans la suite du compte-rendu nous nous limiterons à un champ de force nul.}\\
    \itemb Le nombre de Reynolds $\Re$ :
    $$[\Re] = \varnothing,$$
    $$\Re \in \R^+_*.$$
    \emph{Le nombre de Reynolds est un nombre qui permet de caractériser un écoulement. Si deux fluides différents sont 
    etudié dans la même géométrie et avec la même valeur de $\Re$ l'écoulement est le même.}
  \end{itemize}

\paragraph*{}
  Grandeurs mésoscopiques apparaissent dans la définition des LBM et sont les suivantes :
  \begin{itemize} \label{eq:defmeso}
    \itemb La vitesse des particules $\v$ :
    $$[\v] =  L \cdot T^{-1},$$
    $$\v \in \R^D$$
    \emph{Vitesse des particules à différentier de $\u$ la vitesse macroscopique.}\\
    \itemb La densité de probabilité $f$ :
    $$[f] =  \varnothing,$$
    $$f(\r,\v,t) : \R^{2D+1} \to \R^+.$$
    \emph{La densité de probabilité ne possède pas de dimensions néanmoins intégré sur un volume de 
    \href{https://fr.wikipedia.org/wiki/Espace_des_phases}{l'espace des phase} elle est proportionnelle a la densité et
    donc à la masse volumique $\rho$.}\\
    \itemb L'opérateur collision $C(f)$ :
    $$[C(f)] = \varnothing,$$
    $$C(f) : \R^+ \to \R^+.$$
    \emph{L'opérateur collision est une fonction qui décris l'interaction des particules entre elles, dans le cas des  
    LBM l'opérateur le plus courant (et celui que nous allons utiliser) est 
    \href{https://fr.wikipedia.org/wiki/M\%C3\%A9thode_de_Bhatnagar-Gross-Krook}{l'opérateur de Bhatnagar-Gross-Krook}
    (abrége en BGK\footnote{C'est pour cela que certains auteurs nomment les LBM basées sur cet opérateur des LBGK.})}.
    \\
    \itemb La densité de probabilité à l'équilibre $\feq$ :
    \emph{correspond à la densité de probabilité à l'équilibre étant donné les grandeurs macroscopiques $\u, \rho$. 
    Il apparait dans l'opérateur collision BGK.}\\
    \itemb La vitesse du son dans le fluide $c_s$ :
    $$[c_s] = L\cdot T^{-1},$$
    $$c_s \in \R^+_*.$$
    \emph{Vitesse du son dans le fluide (c'est aussi la vitesse moyenne des particules). Il apparait dans le calcul de 
    $\feq$.}\\
    \itemb coefficient de relaxation $\tau$ :
    $$[\tau] = \varnothing,$$
    $$\tau \in \left[\frac{1}{2},+ \infty \right[.$$
    \emph{Dépend de $\nu, c_s$ et du pas de temps il apparait dans l'opérateur collision BGK.}\\
  \end{itemize}
 
\paragraph*{}
  Pour passer de l'équation de Boltzmann aux LBM il va falloir discrétiser certaines grandeurs, voici les grandeurs
  discrétisées pour la simulation :
  \begin{itemize} \label{eq:defsim}
    \itemb Le pas de discrétisation spatial $\dl$ :
    $$[\dl] = L$$
    $$\dl = \R^+_*.$$
    \emph{On discrétise les valeurs que peuvent prendre $\r$.}\\
    \itemb Le pas de discrétisation temporel $\dt$ :
    $$[\dt] = T$$
    $$\dt = \R^+_*.$$
    \emph{On discrétise les valeurs que peuvent prendre $t$.}\\
    \itemb Le jeu de vitesses $\ei$ :
    $$[\ei] = L \cdot T^{-1}$$
    $$\ei \in \R.$$
    \emph{On discrétise les valeurs que peuvent prendre $\v$ au total on choisiras $Q$ vitesses. Toutes les LBM sont 
    définies par leur dimension nombre spatiales $D$ et leur nombre de vitesses microscopiques $Q$.}\\
    \itemb La densité de probabilité discrétisée $\f$ :
    $$[\f] =  \varnothing,$$
    $$\dim(\f) = D+1.$$
    \emph{La densité de probabilité et désormais discretisée dans
    \href{https://fr.wikipedia.org/wiki/Espace_des_phases}{l'espace des phase}, c'est donc un tableau de taille finie, 
    qui dépend de notre simulation. $\feq$ est discretisé de la même manière.}\\
    \itemb La vitesse de maille $c$ :
    $$[c] = L\cdot T^{-1},$$
    $$c \in \R^+_*.$$
    \emph{Correspond à la vitesse nécessaire pour ce déplacer d'une maille à une autre en un temps $\dt$.}\\
    \itemb La densitée $\rho$ :
    $$[\rho] = \varnothing,$$
    $$\rho : \R^D \to \R^+.$$
    \emph{$\rho$ est désormais adimensionné et vaut la densité du fluide en chaque point.}\\
    \itemb La probabilité à l'équilibre $\w$ :
    $$[\w] = \varnothing,$$
    $$\w \in \R^+_*.$$
    \emph{$\w$ correspond à la probabilité associée à une vitesse $\ei$ pour une vitesse macroscopique $\u = \vec{0}$.}\\
  \end{itemize}