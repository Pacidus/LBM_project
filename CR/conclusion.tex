Dans le cadre de ce projet, nous avons pu implémenter la méthode de Boltzmann sur réseau, ce fut une aventure très riche
et surtout très intéressante.
Les résultats obtenus prouvent la polyvalence de la méthode pour simuler de nombreux systèmes.
La méthode lattice Boltzmann ne bénéficie pas encore d'une documentation claire, ni standardisée, ni rigoureuse\footnote{En tout cas si elle existe elle n'est pas disponible.}  
ce qui peut être très frustrant quand on cherche de la documentation sur le sujet. Il faut garder en tête que c'est un domaine relativement récent. Est il constitue un domaine de recherche très actif.
Les LBM peuve implémenter bien plus que seulement la physique des fluide et nous aurions aimées implémenter de la thermodynamiques et des interactions entre deux fluide mais le temps et la contingence ne nous ont pas permis d'aller aussi loin que souhaité mais la fin de ce projet ne marque pas la fin de ce travail. le \href{https://github.com/Pacidus/LBM_project}{github associé à ce projet} resteras actif encore pour un moment, avec même un \href{https://pacidus.github.io/LBM_project/}{site web associé} en cours de développement. 